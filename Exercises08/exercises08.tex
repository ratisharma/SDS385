%----------------------------------------------------------------------------------------
%	PACKAGES AND OTHER DOCUMENT CONFIGURATIONS
%----------------------------------------------------------------------------------------

\documentclass{article}

\usepackage{fancyhdr} % Required for custom headers
\usepackage{lastpage} % Required to determine the last page for the footer
\usepackage{extramarks} % Required for headers and footers
\usepackage[usenames,dvipsnames]{color} % Required for custom colors
\usepackage{graphicx} % Required to insert images
\usepackage{amsmath}
\usepackage{listings} % Required for insertion of code
\usepackage[]{algorithm2e}
%\usepackage{couriernew} % Required for the courier font

\usepackage{enumerate} % Required for enumerating with letters

\usepackage{mathpazo}
% Palatino
\usepackage{avant}
\usepackage{inconsolata}

\DeclareMathOperator*{\argmin}{arg\,min}
\DeclareMathOperator*{\sign}{sign}
\DeclareMathOperator*{\prox}{prox}

\newcommand{\norm}[1]{\left\lVert#1\right\rVert}

% Margins
\topmargin=-0.45in
\evensidemargin=0in
\oddsidemargin=0in
\textwidth=6.5in
\textheight=9.0in
\headsep=0.25in

\linespread{1.1} % Line spacing

% Set up the header and footer
\pagestyle{fancy}
\lhead{\hmwkAuthorName} % Top left header
\chead{\hmwkClass\ : \hmwkTitle} % Top center head
\rhead{} % Top right header
\lfoot{\lastxmark} % Bottom left footer
\cfoot{} % Bottom center footer
\rfoot{Page\ \thepage\ of\ \protect\pageref{LastPage}} % Bottom right footer
\renewcommand\headrulewidth{0.4pt} % Size of the header rule
\renewcommand\footrulewidth{0.4pt} % Size of the footer rule

\setlength\parindent{0pt} % Removes all indentation from paragraphs

%----------------------------------------------------------------------------------------
%	CODE INCLUSION CONFIGURATION
%----------------------------------------------------------------------------------------

\definecolor{MyDarkGreen}{rgb}{0.0,0.4,0.0} % This is the color used for comments
\lstloadlanguages{R} % Load R syntax for listings, for a list of other languages supported see: ftp://ftp.tex.ac.uk/tex-archive/macros/latex/contrib/listings/listings.pdf
\lstset{language=R, % Use R in this example
        frame=single, % Single frame around code
        basicstyle=\small\ttfamily, % Use small true type font
        keywordstyle=[1]\color{Blue}, % Perl functions bold and blue
        keywordstyle=[2]\color{Purple}, % Perl function arguments purple
        keywordstyle=[3]\color{Blue}\underbar, % Custom functions underlined and blue
        identifierstyle=, % Nothing special about identifiers                                         
        commentstyle=\usefont{T1}{pcr}{m}{sl}\color{MyDarkGreen}\small, % Comments small dark green courier font
        stringstyle=\color{Purple}, % Strings are purple
        showstringspaces=false, % Don't put marks in string spaces
        tabsize=4, % 5 spaces per tab
        %
        % Put standard Perl functions not included in the default language here
        morekeywords={rand},
        %
        % Put Perl function parameters here
        morekeywords=[2]{on, off, interp},
        %
        % Put user defined functions here
        morekeywords=[3]{test},
       	%
        morecomment=[l][\color{Blue}]{...}, % Line continuation (...) like blue comment
        numbers=left, % Line numbers on left
        firstnumber=1, % Line numbers start with line 1
        numberstyle=\tiny\color{Blue}, % Line numbers are blue and small
        stepnumber=5 % Line numbers go in steps of 5
}

% Creates a new command to include a perl script, the first parameter is the filename of the script (without .pl), the second parameter is the caption
\newcommand{\rscript}[2]{
\begin{itemize}
\item[]\lstinputlisting[caption=#2,label=#1]{#1.r}
\end{itemize}
}

%----------------------------------------------------------------------------------------
%	DOCUMENT STRUCTURE COMMANDS
%	Skip this unless you know what you're doing
%----------------------------------------------------------------------------------------

% Header and footer for when a page split occurs within a problem environment
\newcommand{\enterProblemHeader}[1]{
\nobreak\extramarks{#1}{#1 continued on next page\ldots}\nobreak
\nobreak\extramarks{#1 (continued)}{#1 continued on next page\ldots}\nobreak
}

% Header and footer for when a page split occurs between problem environments
\newcommand{\exitProblemHeader}[1]{
\nobreak\extramarks{#1 (continued)}{#1 continued on next page\ldots}\nobreak
\nobreak\extramarks{#1}{}\nobreak
}

\setcounter{secnumdepth}{0} % Removes default section numbers
\newcounter{homeworkProblemCounter} % Creates a counter to keep track of the number of problems

\newcommand{\homeworkProblemName}{}
\newenvironment{homeworkProblem}[1][Problem \arabic{homeworkProblemCounter}]{ % Makes a new environment called homeworkProblem which takes 1 argument (custom name) but the default is "Problem #"
\stepcounter{homeworkProblemCounter} % Increase counter for number of problems
\renewcommand{\homeworkProblemName}{#1} % Assign \homeworkProblemName the name of the problem
\section{\homeworkProblemName} % Make a section in the document with the custom problem count
\enterProblemHeader{\homeworkProblemName} % Header and footer within the environment
}{
\exitProblemHeader{\homeworkProblemName} % Header and footer after the environment
}

\newcommand{\problemAnswer}[1]{ % Defines the problem answer command with the content as the only argument
\noindent\framebox[\columnwidth][c]{\begin{minipage}{0.98\columnwidth}#1\end{minipage}} % Makes the box around the problem answer and puts the content inside
}

\newcommand{\homeworkSectionName}{}
\newenvironment{homeworkSection}[1]{ % New environment for sections within homework problems, takes 1 argument - the name of the section
\renewcommand{\homeworkSectionName}{#1} % Assign \homeworkSectionName to the name of the section from the environment argument
\subsection{\homeworkSectionName} % Make a subsection with the custom name of the subsection
\enterProblemHeader{\homeworkProblemName\ [\homeworkSectionName]} % Header and footer within the environment
}{
\enterProblemHeader{\homeworkProblemName} % Header and footer after the environment
}

%----------------------------------------------------------------------------------------
%	NAME AND CLASS SECTION
%----------------------------------------------------------------------------------------

\newcommand{\hmwkTitle}{Exercises 8 - Spacial smoothing at scale} % Assignment title
\newcommand{\hmwkDueDate}{\today} % Due date
\newcommand{\hmwkClass}{SDS\ 385} % Course/class
\newcommand{\hmwkClassTime}{} % Class/lecture time
\newcommand{\hmwkClassInstructor}{Professor Scott} % Teacher/lecturer
\newcommand{\hmwkAuthorName}{Spencer Woody} % Your name

\newcommand{\yhat}{\hat{y}}
\newcommand{\E}{\text{E}}

%----------------------------------------------------------------------------------------
%	TITLE PAGE
%----------------------------------------------------------------------------------------

\title{
\vspace{2in}
\textmd{\textbf{\hmwkClass:\ \hmwkTitle}}\\
\normalsize\vspace{0.1in}\small{\hmwkDueDate}\\
\vspace{0.1in}\large{\textit{\hmwkClassInstructor\ }}
\vspace{3in}
}

\author{\textbf{\hmwkAuthorName}}
\date{} % Insert date here if you want it to appear below your name

%----------------------------------------------------------------------------------------

\begin{document}

\maketitle

\newpage


%----------------------------------------------------------------------------------------
%	PROBLEM 1
%----------------------------------------------------------------------------------------

\begin{homeworkProblem}
\large
\textbf{Laplacian smoothing}
\normalsize

Normalsize text

\end{homeworkProblem}

%----------------------------------------------------------------------------------------
%	PROBLEM 2
%----------------------------------------------------------------------------------------

\begin{homeworkProblem}
\large
\textbf{Graph fused lasso}
\normalsize

Normalsize text

\end{homeworkProblem}

%----------------------------------------------------------------------------------------
%	PROBLEM 3
%----------------------------------------------------------------------------------------

% To have just one problem per page, simply put a \clearpage after each problem

\begin{homeworkProblem}
\large
\textbf{ADMM}
\normalsize

We use ADMM to form Lasso estimates. Let $A$ be our feature matrix, and $x$ is our coefficient vector, and $b$ is the response vector. 

\begin{align}
	\min &\frac{1}{2} \norm{Ax - b}^2_2 + \lambda \norm{z}_1 \\
	\text{s.t. } &x - z = 0
\end{align}
%
%
%
The Lagrangian is 
%
%
%
\begin{align}
	L_p(x, z, u) &= \frac{1}{2} \norm{Ax - b}^2_2 + \lambda \norm{z}_1 + u^T(x-z) + \frac{\rho}{2} \norm{x - z}^2_2.
\end{align}
%
%
%
\begin{align}
	x^{k+1} &= \argmin_x L_p(x, z^k, u^k) \\
	&= \argmin_x \left\{ \frac{1}{2} \norm{Ax - b}^2_2 + \lambda \norm{z^k}_1 + (u^k)^T(x-z) + \frac{\rho}{2} \norm{x - z^k}^2_2 \right\}
\end{align}
%
%
%
We take the gradient of the objective function with respect to $x$ and set it equal to 0,
%
%
%
\begin{align}
	\nabla_x L_p(x, z, u) &= A^T(Ax^{k+1} - b) + u^k + \rho(x^{k+1} - z^k) \\
	&= (A^TA + \rho I)x^{k+1} - (A^Tb + \rho z^k - u^k) = 0 \\
	x^{k+1} &= (A^TA + \rho I)^{-1} (A^Tb + \rho z^k - u^k) \text{,  let } v^k = u^k / \rho\\
	x^{k+1} &= (A^TA + \rho I)^{-1} (A^Tb + \rho (z^k - v^k))
\end{align}
%
%
%
\begin{align}
	z^{k+1} &= \argmin_z L_p(x^{k+1}, z, u^k) \\
	&= \argmin_z \left\{ \frac{1}{2} \norm{Ax^{k+1} - b}^2_2 + \lambda \norm{z}_1 + (u^k)^T(x^{k+1}-z) + \frac{\rho}{2} \norm{x^{k+1} - z}^2_2 \right\} \\
	&= \argmin_z \left\{ \frac{\lambda}{\rho} \norm{z}_1 + (v^k)^T(x^{k+1}-z) + \frac{1}{2} \norm{x^{k+1} - z}^2_2 \right\} \\
	&= \argmin_z \left\{ \frac{\lambda}{\rho} \norm{z}_1 - (z - x^{k+1})^Tv^k + \frac{1}{2} \norm{z - x^{k+1}}^2_2 \right\} \\
	&= \prox_{\gamma = 1} \frac{\lambda}{\rho} \norm{w^k}\text{ ,  }w^k = x^{k+1} + v^k
\end{align}
%
%
%
So $z^{k+1}$ is subjected to the soft threshholding from the previous two exercises. Finally we update $u^k$, and, identically, $v^k$.
%
%
%
\begin{align}
	u^{k+1} &= u^k + \rho(x^{k+1} - z^{k+1}) \\
	\rho v^{k+1} &= \rho v^k + \rho(x^{k+1} - z^{k+1}) \\
	v^{k+1} &= v^k + x^{k+1} - z^{k+1} 
\end{align}
%
%
%
With $\lambda =  0.002420476$, the regular proximal gradient method takes 5887 iterations, the accelerated proximal gradient method takes 626 iterations, and the ADMM takes 146 iterations. See Figure \ref{fig:compplot}.
%
%
%
\begin{figure}[htbp]
	\centering
%		\includegraphics[scale=0.75]{compplot.pdf}
	\caption{Comparison of convergence for different methods}
	\label{fig:compplot}
\end{figure}
%
%
%
%
\end{homeworkProblem}



%%----------------------------------------------------------------------------------------

% \pagebreak
% R script \texttt{linesearch.R}
% \lstinputlisting[language=R]{linesearch.R}
%
\pagebreak
\textsf{R} script for \texttt{myfuns.R}
%\lstinputlisting[language=R]{myfuns.R}

\pagebreak
\textsf{R} script for \texttt{e6.R}
%\lstinputlisting[language=R]{e6.R}

%----------------------------------------------------------------------------------------

\end{document}