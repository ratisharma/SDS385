%----------------------------------------------------------------------------------------
%	PACKAGES AND OTHER DOCUMENT CONFIGURATIONS
%----------------------------------------------------------------------------------------

\documentclass{article}

\usepackage{fancyhdr} % Required for custom headers
\usepackage{lastpage} % Required to determine the last page for the footer
\usepackage{extramarks} % Required for headers and footers
\usepackage[usenames,dvipsnames]{color} % Required for custom colors
\usepackage{graphicx} % Required to insert images
\usepackage{listings} % Required for insertion of code
%\usepackage{couriernew} % Required for the courier font
\usepackage{enumerate} % Required for enumerating with letters
\usepackage{amsmath}
\usepackage{amssymb}
\usepackage{algorithm}
\usepackage[noend]{algpseudocode}
\usepackage[thinlines]{easytable}

\DeclareMathOperator*{\argmin}{arg\,min}


% Margins
\topmargin=-0.45in
\evensidemargin=0in
\oddsidemargin=0in
\textwidth=6.5in
\textheight=9.0in
\headsep=0.25in

\linespread{1.1} % Line spacing

% Set up the header and footer
\pagestyle{fancy}
\lhead{\hmwkAuthorName} % Top left header
\chead{\hmwkClass\ : \hmwkTitle} % Top center head
\rhead{} % Top right header
\lfoot{\lastxmark} % Bottom left footer
\cfoot{} % Bottom center footer
\rfoot{Page\ \thepage\ of\ \protect\pageref{LastPage}} % Bottom right footer
\renewcommand\headrulewidth{0.4pt} % Size of the header rule
\renewcommand\footrulewidth{0.4pt} % Size of the footer rule

\setlength\parindent{0pt} % Removes all indentation from paragraphs

%----------------------------------------------------------------------------------------
%	CODE INCLUSION CONFIGURATION
%----------------------------------------------------------------------------------------

\definecolor{MyDarkGreen}{rgb}{0.0,0.4,0.0} % This is the color used for comments
\lstloadlanguages{R} % Load R syntax for listings, for a list of other languages supported see: ftp://ftp.tex.ac.uk/tex-archive/macros/latex/contrib/listings/listings.pdf
\lstset{language=R, % Use R in this example
        frame=single, % Single frame around code
        basicstyle=\small\ttfamily, % Use small true type font
        keywordstyle=[1]\color{Blue}, % Perl functions bold and blue
        keywordstyle=[2]\color{Purple}, % Perl function arguments purple
        keywordstyle=[3]\color{Blue}\underbar, % Custom functions underlined and blue
        identifierstyle=, % Nothing special about identifiers                                         
        commentstyle=\usefont{T1}{pcr}{m}{sl}\color{MyDarkGreen}\small, % Comments small dark green courier font
        stringstyle=\color{Purple}, % Strings are purple
        showstringspaces=false, % Don't put marks in string spaces
        tabsize=4, % 5 spaces per tab
        %
        % Put standard Perl functions not included in the default language here
        morekeywords={rand},
        %
        % Put Perl function parameters here
        morekeywords=[2]{on, off, interp},
        %
        % Put user defined functions here
        morekeywords=[3]{test},
       	%
        morecomment=[l][\color{Blue}]{...}, % Line continuation (...) like blue comment
        numbers=left, % Line numbers on left
        firstnumber=1, % Line numbers start with line 1
        numberstyle=\tiny\color{Blue}, % Line numbers are blue and small
        stepnumber=5 % Line numbers go in steps of 5
}

% Creates a new command to include a perl script, the first parameter is the filename of the script (without .pl), the second parameter is the caption
\newcommand{\rscript}[2]{
\begin{itemize}
\item[]\lstinputlisting[caption=#2,label=#1]{#1.r}
\end{itemize}
}

%----------------------------------------------------------------------------------------
%	DOCUMENT STRUCTURE COMMANDS
%	Skip this unless you know what you're doing
%----------------------------------------------------------------------------------------

% Header and footer for when a page split occurs within a problem environment
\newcommand{\enterProblemHeader}[1]{
\nobreak\extramarks{#1}{#1 continued on next page\ldots}\nobreak
\nobreak\extramarks{#1 (continued)}{#1 continued on next page\ldots}\nobreak
}

% Header and footer for when a page split occurs between problem environments
\newcommand{\exitProblemHeader}[1]{
\nobreak\extramarks{#1 (continued)}{#1 continued on next page\ldots}\nobreak
\nobreak\extramarks{#1}{}\nobreak
}

\setcounter{secnumdepth}{0} % Removes default section numbers
\newcounter{homeworkProblemCounter} % Creates a counter to keep track of the number of problems

\newcommand{\homeworkProblemName}{}
\newenvironment{homeworkProblem}[1][Problem \arabic{homeworkProblemCounter}]{ % Makes a new environment called homeworkProblem which takes 1 argument (custom name) but the default is "Problem #"
\stepcounter{homeworkProblemCounter} % Increase counter for number of problems
\renewcommand{\homeworkProblemName}{#1} % Assign \homeworkProblemName the name of the problem
\section{\homeworkProblemName} % Make a section in the document with the custom problem count
\enterProblemHeader{\homeworkProblemName} % Header and footer within the environment
}{
\exitProblemHeader{\homeworkProblemName} % Header and footer after the environment
}

\newcommand{\problemAnswer}[1]{ % Defines the problem answer command with the content as the only argument
\noindent\framebox[\columnwidth][c]{\begin{minipage}{0.98\columnwidth}#1\end{minipage}} % Makes the box around the problem answer and puts the content inside
}

\newcommand{\homeworkSectionName}{}
\newenvironment{homeworkSection}[1]{ % New environment for sections within homework problems, takes 1 argument - the name of the section
\renewcommand{\homeworkSectionName}{#1} % Assign \homeworkSectionName to the name of the section from the environment argument
\subsection{\homeworkSectionName} % Make a subsection with the custom name of the subsection
\enterProblemHeader{\homeworkProblemName\ [\homeworkSectionName]} % Header and footer within the environment
}{
\enterProblemHeader{\homeworkProblemName} % Header and footer after the environment
}





%----------------------------------------------------------------------------------------
%	NAME AND CLASS SECTION
%----------------------------------------------------------------------------------------

\newcommand{\hmwkTitle}{Exercises 1 - Preliminaries} % Assignment title
\newcommand{\hmwkDueDate}{August\ 23,\ 2016} % Due date
\newcommand{\hmwkClass}{SDS\ 385} % Course/class
\newcommand{\hmwkClassTime}{} % Class/lecture time
\newcommand{\hmwkClassInstructor}{Professor James Scott} % Teacher/lecturer
\newcommand{\hmwkAuthorName}{Spencer Woody} % Your name

%----------------------------------------------------------------------------------------
%	TITLE PAGE
%----------------------------------------------------------------------------------------

\title{
\vspace{2in}
\textmd{\textbf{\hmwkClass:\ \hmwkTitle}}\\
\normalsize\vspace{0.1in}\small{\hmwkDueDate}\\
\vspace{0.1in}\large{\textit{\hmwkClassInstructor\ }}
\vspace{3in}
}

\author{\textbf{\hmwkAuthorName}}
\date{} % Insert date here if you want it to appear below your name

%----------------------------------------------------------------------------------------

\begin{document}

\maketitle

\newpage

%----------------------------------------------------------------------------------------
%	PROBLEM 1
%----------------------------------------------------------------------------------------

% To have just one problem per page, simply put a \clearpage after each problem

\begin{homeworkProblem}
\begin{enumerate}[(A)]
%%
%%
%%
	\item %a
		\begin{align}
			\hat{\beta} &= \argmin_{\beta \in \mathbb{R}^p} \sum_{i=1}^{N} \frac{w_i}{2} \left ( y_i - x_i^T\beta \right )^2 \\
			&= \argmin_{\beta \in \mathbb{R}^p} \frac{1}{2} (Y-X \beta)^T W (Y-X\beta)
		\end{align}
		\begin{align}
			\frac{1}{2} (Y-X \beta)^T W (Y-X\beta) &= \frac{1}{2} (Y^T - \beta^T X^T)W(Y-X\beta) \\
			&= \frac{1}{2} (Y^TW - \beta^T X^TW)(Y-X\beta) \\
			&= \frac{1}{2} (Y^TWY -\beta^TX^TWY-Y^TWX\beta + \beta^TX^TWX\beta) \\
			&= \frac{1}{2} (Y^TWY - 2(X\beta)^TWY +\beta^TX^TWX\beta) \\
			&= \frac{1}{2} Y^TWY - (X\beta)^TWY + \frac{1}{2} \beta^TX^TWX\beta, 
		\end{align}
		because 
		\begin{align}
			\beta^TX^TWY = (X\beta)^TWY,
		\end{align}
		and
		\begin{align}
			Y^TWX\beta &= (Y^TWX\beta)^T \because Y^TWX\beta \in \mathbb{R}^1 \\
			(Y^TWX\beta)^T &= (WX\beta)^TY = (X\beta)^TW^TY = (X\beta)^TWY.
		\end{align}
		We want to minimize the objective function from Eqn. (7), so we take the gradient with respect to $\beta$ and set it equal to zero. For each of the three terms, their are respective gradients with respect to $\beta$ are 
		\begin{enumerate}[(i)]
			\item % i
				\begin{align}
					\frac{\partial}{\partial \beta} \frac{1}{2} Y^TWY = 0
				\end{align}
			\item % ii
				\begin{align}
					\frac{\partial}{\partial \beta} -(X\beta)^TWY = - X^TWY
				\end{align}
			\item % iii
				\begin{align}
					\frac{\partial}{\partial \beta} \frac{1}{2} \beta^TX^TWX\beta &= \frac{1}{2} \beta^T (X^TWX + (X^TWX)^T) \\
					&= X^TWX\beta.
				\end{align}
		\end{enumerate}
		Summing these terms and equaling them to zero yields
		\begin{align}
			X^TWX\beta - X^TWY &= 0 \therefore \\
			 (X^TWX)\hat{\beta} &= X^TWY
		\end{align}
%%
%%
%%
	\item %b
		The brute force method of solving Eqn. (16) is the \emph{inversion method}, i.e.
		\begin{align}
			\hat{\beta} = (X^TWX)^{-1}X^TWy.
		\end{align} 
		However, this method is computationally expensive. Therefore I propose an alternative methods to solving this matrix equation using the Cholesky decomposition.
			\textbf{Cholesky Decomposition} \\
			Let 
			\begin{align}
				C = X^TWX, \;\;D=X^TWy
			\end{align}
			so 
			\begin{align}
				C\hat{\beta} = D.
			\end{align}
			We decompose matrix $C$ into a product of a lower-triangular matrix and an upper-triangular matrix, such that $U=L^T$ so
			\begin{align}
				C &= LU = LL^T \therefore \\
				LL^T\hat{\beta} &= D.
			\end{align}
			Furthermore we define matrix $A = L^T\hat{\beta}$. Thus we are left with two matrix equations to solve. 
			\begin{align}
				LA &= D \\
				L^T\hat{\beta} &= A
			\end{align}
			This method will be much less computationally intensive than the inversion method because of the fact that the two left-matrices $L$ and $U=L^T$ are triangular. We still must invert $L$ and $L^T$ but this is simpler than taking an inverse of a more complicated matrix $X^TWX$. This is similar to an LU decomposition, with the exception that we necessarily have two triangular matrices that are transposes of one another. Therefore, this method gains a computational advantage over LU decomposition from symmetric exploitation.
	\item %c
	Code for implementing this method is shown in the appendix to this paper.
	\item %d
	The Matrix package within R is suited to handle sparse matrices with the \texttt{sparse} argument within the \texttt{Matrix} command. 
\end{enumerate}
\end{homeworkProblem}


%
%%----------------------------------------------------------------------------------------
%%	PROBLEM 2
%%----------------------------------------------------------------------------------------

% \frac{1}{1+\exp(-x_i^T\beta)}
% \frac{\exp(-x_i^T\beta)}{1+\exp(-x_i^T\beta)}

\pagebreak

\begin{homeworkProblem}
\begin{enumerate}[(A)]
%%
%%
%%
\item % A
	We have $y_i \sim \text{Binomial}(m_i, w_i)$, where 
	\begin{align}
		w_i = \frac{1}{1+\exp(-x_i^T\beta)}, \;\; 1-w_i = \frac{\exp(-x_i^T\beta)}{1+\exp(-x_i^T\beta)},
	\end{align}
	so the negative log likelihood is
	\begin{align}
		\ell(\beta) &= -\log \left \{ \prod_{i=1}^N p(y_i | \beta)  \right \} \\
		&= -\log \left \{ \prod_{i=1}^N \binom {m_i}{y_i}(w_i)^{y_i}(1-w_i)^{m_i-y_i}  \right \} \\
		&= - \left \{ \sum_{i=1}^{N} \left ( \log\binom {m_i}{y_i} + y_i \log(w_i) + (m_i-y_i)\log(1-w_i) \right ) \right \} \\
		&= - \left \{ \sum_{i=1}^{N} \left ( \log\binom {m_i}{y_i} + y_i \log\left ( \frac{1}{1+\exp(-x_i^T\beta)} \right ) + (m_i-y_i)\log\left ( \frac{\exp(-x_i^T\beta)}{1+\exp(-x_i^T\beta)} \right) \right ) \right \} \\
		& = - \left \{ \sum_{i=1}^{N} \left ( \log\binom {m_i}{y_i} - y_i\log(1+\exp(-x_i^T\beta)) - (m_i-y_i)x_i^T\beta -m_i\log(1+\exp(-x_i^T\beta)) + y_i\log(1+\exp(-x_i^T\beta)) \right ) \right \} \\
		& = - \left \{ \sum_{i=1}^{N} \left ( \log\binom {m_i}{y_i} - (m_i-y_i)x_i^T\beta -m_i\log(1+\exp(-x_i^T\beta)) \right ) \right \} \\
		& = \sum_{i=1}^{N} \left ((m_i-y_i)x_i^T\beta +m_i\log(1+\exp(-x_i^T\beta)) - \log\binom {m_i}{y_i} \right ) \\
		%%%
		%%%
		%%%
	\end{align}
	The gradient for this expression is, 
	\begin{align}
		\nabla \ell (\beta) &= \sum_{i=1}^N \left ( (m_i-y_i)x_i - m_i \frac{\exp({-x_i^T\beta})}{1+\exp(-x_i^T\beta)}x_i \right ) \\
		&= \sum_{i=1}^N \left ( (m_i-y_i)x_i - m_i (1-w_i)x_i \right ) \\
		&= \sum_{i=1}^N (m_iw_i-y_i)x_i \\
		&= -X^T(y-mw)
	\end{align}
	where $y$ is the $n \times 1$ vector of responses and $mw$ is the element-wise product of the two $n \times 1$ vectors $m$ and $w$.
%%
%%
%%
%%
%%
\item % B
	Code for implementing the gradient descent method is shown in the appendix. Note that we normalize the values in the $X$ matrix and add a column of 1's to make an intercept term. We start by having an initial arbitrary guess for $\beta$, which we define as $\beta_0$. Then we use an iterative process to converge upon the true value of $\beta$ based on the calculated gradient of the log likelihood at $\hat{\beta}_t$ and an arbitrary step size, $\alpha$ as follows
	\begin{align}
		\hat{\beta}_{t+1} = \hat{\beta}_t - \alpha \times \nabla\ell(\hat{\beta}_t)
	\end{align}
	We use an intial guess of $\beta_0 = 0$, a step size of $\alpha = 0.025$, and 50,000 iterations and reach convergence in optimizing the log likelihood, as shown in the trace plot below. Our final estimations of ${\beta}$ are reported below along with estimations from R's native \texttt{glm} function. The two sets of estimates are in close agreement with one another.
		\begin{figure}[htp!]
		\centering
			\includegraphics[scale=0.5]{beta_trace1.png}
		\end{figure}
		% Table
		\begin{table}[]
		\centering
		\label{my-label}
		\begin{tabular}{l|r|r}
		                        & Grad descent & R: \texttt{glm} \\ \hline
								\\[-1em]
		$\hat{\beta}_1$      & 0.48553           & 0.48702         \\  \hline
		\\[-1em]
		$\hat{\beta}_2$      & -7.14618          & -7.22185         \\ \hline
		\\[-1em]
		$\hat{\beta}_3$      & 1.65481           & 1.65476         \\ \hline
		\\[-1em]
		$\hat{\beta}_4$      & -1.80713          & -1.73763        \\ \hline
		\\[-1em]
		$\hat{\beta}_5$      & 13.99290          & 14.00485        \\ \hline
		\\[-1em]
		$\hat{\beta}_6$      & 1.07426           & 1.07495         \\ \hline
		\\[-1em]
		$\hat{\beta}_7$      & -0.07319          & -0.07723        \\ \hline
		\\[-1em]
		$\hat{\beta}_8$      & 0.67573           & 0.67512         \\ \hline
		\\[-1em]
		$\hat{\beta}_9$      & 2.59383           & 2.59287         \\ \hline
		\\[-1em]
		$\hat{\beta}_{10}$ & 0.44615           & 0.44626         \\ \hline
		\\[-1em]
		$\hat{\beta}_{11}$ & -0.48276          & -0.48248           
		\end{tabular}
		\caption{Comparison of results from gradient descent and \texttt{glm}}
		\end{table}
%%
%%
%%
%%
%%
\item % C
	We need to calculate the Hessian matrix of the log likelihood function, $\nabla^2(\ell(\beta))$. The Hessian will be a $P\times P$ matrix, with the element in row $i$ and column $j$ being\footnote{Notice the reindexing shown below for summations.} 
	\begin{align}
		\frac{\partial^2}{\partial \beta_i \partial \beta _j}\ell (\beta) &= \frac{\partial}{\partial \beta_i} \left ( \frac{\partial}{\partial \beta_j} \ell (\beta) \right ) \\
		&= \frac{\partial}{\partial \beta_i} \left ( \frac{\partial}{\partial \beta_j} \sum_{k=1}^N(\ldots) \right ) \\
		&= \frac{\partial}{\partial \beta_i} \left ( \sum_{k=1}^N(m_kw_k-y_k)x_{kj} \right )  \\
		&= \sum_{k=1}^{N} x_{ki}x_{kj}m_k w_k(1-w_k)
	\end{align}
	Note: 
	\begin{align}
		\frac{\partial}{\partial \beta_i} w_k &= x_{ki} \frac{\exp(-x_k^T\beta)}{(1+\exp(-x_k^T\beta))^2} \\
		&= x_{ki} w_k(1-w_k)
	\end{align}
	This matrix is equivalent to $X^TWX$ where $W = \text{diag}(m_1w_1(1-w_1),\ldots, m_Nw_N(1-w_N))$ \\
%%
%%
%%
%%
%%
\item % D
Now we use Newton's to estimate $\beta$. This is also an iterative process, though now we need far fewer iterations to achieve convergence because we are taking the curvature of our objective function ($\ell(\beta)$) into account. In fact, we only use 10 iterations and achieve estimates $\hat{\beta}$ which are \emph{exactly} in line with estimates from \texttt{glm}. \\
\textbf{Newton's Method:} 
\begin{align}
	\hat{\beta}_{t+1} = \hat{\beta}_t - (\nabla^2\ell(\hat{\beta}_t))^{-1}\nabla\ell(\hat{\beta_t})
\end{align}
\begin{table}[htp!]
		\centering
		\label{my-label2}
		\begin{tabular}{l|r|r}
		                        & N.'s method & R: \texttt{glm} \\ \hline
								\\[-1em]
		$\hat{\beta}_1$      & 0.48702           & 0.48702         \\  \hline
		\\[-1em]
		$\hat{\beta}_2$      & -7.22185          & -7.22185         \\ \hline
		\\[-1em]
		$\hat{\beta}_3$      & 1.65476          & 1.65476         \\ \hline
		\\[-1em]
		$\hat{\beta}_4$      & -1.73763           & -1.73763        \\ \hline
		\\[-1em]
		$\hat{\beta}_5$      & 14.00485        & 14.00485        \\ \hline
		\\[-1em]
		$\hat{\beta}_6$      & 1.07495             & 1.07495         \\ \hline
		\\[-1em]
		$\hat{\beta}_7$      & -0.07723          & -0.07723        \\ \hline
		\\[-1em]
		$\hat{\beta}_8$      & 0.67512            & 0.67512         \\ \hline
		\\[-1em]
		$\hat{\beta}_9$      & 2.59287         & 2.59287         \\ \hline
		\\[-1em]
		$\hat{\beta}_{10}$ & 0.44626          & 0.44626         \\ \hline
		\\[-1em]
		$\hat{\beta}_{11}$ & -0.48248           & -0.48248           
		\end{tabular}
		\caption{Comparison of results from Newton's method and \texttt{glm}}
	\end{table}
%%
%%
%%
%%
%%
\item % E
Gradient descent requires many iterations while Newton's method must invert a matrix, which may either be improssible and is computationally intensive for large matrices.
%%
%%
%%
%%
%%
\end{enumerate}
\end{homeworkProblem}




%%----------------------------------------------------------------------------------------
%%	LIST CODE
%%----------------------------------------------------------------------------------------

\pagebreak
% \rscript{homework03.r}{Sample Perl Script With Highlighting}
\lstinputlisting[language=R]{exercises01.R}

\pagebreak
\lstinputlisting[language=R]{exercises01pt2.R}
%----------------------------------------------------------------------------------------

\end{document}
